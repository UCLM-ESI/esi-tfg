\chapter{Antecedentes}


\section{Una sección}

Para hacer viñetas como ésta (cambiando márgenes u otras propiedades):

\begin{itemize}[noitemsep, label=$\triangleright$]
\item esto es
\item una pequeña
\item muestra
\end{itemize}

se recomienda el paquete
\href{http://mirror.ctan.org/macros/latex/contrib/enumitem/enumitem.pdf}{«enumitem»}
(ya incluido en \texttt{arco-pfc.cls}).


\section{Una figura}

Las figuras se referencian así (ver figura~\ref{fig:informatica}). Recuerda que
no tienen porqué aparecer en el lugar donde se ponen (mira un libro de
verdad). \LaTeX{} las colocará donde mejor queden, No te empeñes en
contradecirle, él sabe mucho de tipografía.

\begin{figure}[!h]
\begin{center}
%\includegraphics[width=0.2\textwidth]{}
\caption{Escudo oficial de informática}
\label{fig:informatica}
\end{center}
\end{figure}

Por cierto, los títulos de tablas, figuras y otro elementos flotantes (los
\texttt{caption}) no deben acabar en punto~\cite{sousa}.


\section{Un cuadro}
\label{sec:uncuadro}

Se denominan «tablas» cuando contienen datos con relaciones numéricas. En
general se denominan «cuadros». Si las columnas están bien alineadas, las líneas
verticales estorban más que ayudan (no las pongas). Los cuadros se referencian
de este modo (ver cuadro~\ref{tab:rpc-semantics}).

\begin{table}[htbp]
  \centering
  {\small
  


\begin{tabular}{p{.2\textwidth}p{.2\textwidth}p{.2\textwidth}p{.2\textwidth}}
  \tabheadformat
  \tabhead{Tipo de fallo}   &
  \tabhead{Sin fallos}      &
  \tabhead{Mensaje perdido} &
  \tabhead{Servidor caído}  \\
\hline
\textit{Maybe}         & Ejecuta:   1 & Ejecuta: 0/1        & Ejecuta: 0/1 \\
                       & Resultado: 1 & Resultado: 0        & Resultado: 0 \\
\hline
\textit{Al-least-once} & Ejecuta:   1 & Ejecuta:   $\geq$ 1 & Ejecuta:   $\geq$ 0 \\
                       & Resultado: 1 & Resultado: $\geq$ 1 & Resultado: $\geq$ 0 \\
\hline
\textit{At-most-once}  & Ejecuta:   1 & Ejecuta:   1        & Ejecuta: 0/1 \\
                       & Resultado: 1 & Resultado: 1        & Resultado: 0 \\
\hline
\textit{Exactly-once}  & Ejecuta:   1 & Ejecuta:   1        & Ejecuta:   1 \\
                       & Resultado: 1 & Resultado: 1        & Resultado: 1 \\
\hline
\end{tabular}


% Local variables:
%   coding: utf-8
%   ispell-local-dictionary: "castellano8"
%   TeX-master: "main.tex"
% End:

  }
  \caption[Semánticas de \acs{RPC} en presencia de distintos fallos]
  {Semánticas de \acs{RPC} en presencia de distintos fallos
    (\textsc{Puder}~\cite{puder05:_distr_system_archit})}
  \label{tab:rpc-semantics}
\end{table}


\section{Un listado de código}
\label{sec:listado}

Puedes referenciar un listado con~\ref{code:hello}. Éste es un listado flotante,
pero también pueden ser «no flotantes» (mira la documentación del paquete
\href{http://www.ctan.org/get/macros/latex/contrib/listings/listings.pdf}{«listings»}).

\begin{listing}[
  float,
  language = C,
  caption  = {«Hola mundo» en C},
  label    = code:hello]
#include <stdio.h>
int main(int argc, char *argv[]) {
    puts("Hola mundo\n");
}
\end{listing}




\section{Citas y referencias cruzadas}

Una cita~\cite{design_patterns} y una referencia a la segunda sección (véase
\S\,\ref{sec:uncuadro}).

Si estás viendo la versión PDF de este documento puedes pinchar la cita o el
número de sección. Son hiper-enlaces que llevan al elemento correspondiente. Todos los
elementos que hacen referencia a otra cosa (figuras, tablas, listados,
secciones, capítulos, citas, páginas web, etc.) pueden ser «pinchables» gracias al
paquete \href{http://latex.tugraz.at/_media/docs/hyperref.pdf}{«hyperref»}.


% Local Variables:
%   coding: utf-8
%   mode: latex
%   mode: flyspell
%   ispell-local-dictionary: "castellano8"
% End:
