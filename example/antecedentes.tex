\chapter{Antecedentes}

Este capítulo incluye unas nociones básicas de \LaTeX{} y algunos consejos
sencillos de composición para sacar todo el jugo a la clase \arcopfc. Ten
presente que este capítulo está pensado para que leas el código fuente y lo
compares con el resultado en PDF.

\section{Estilos de texto}

Debido a su continuo uso, se muestra entre paréntesis la combinación de emacs
para incluir el comando \LaTeX{} correspondiente.

\begin{itemize}[noitemsep]
\item Normal.
\item \textbf{Negrita} (C-c-f-b).
\item \textit{Itálica} (C-c-f-i).
\item \emph{Enfatizada} (C-c-f-e). Fíjate que el estilo que se obtiene al
  enfatizar depende del estilo del texto en el que se incluya: \textit{texto en
    itálica y \emph{enfatizado} en medio}.
\item \texttt{Monoespaciada} (C-c-f-t)
\end{itemize}

Otros de menos uso:

\begin{itemize}[noitemsep]
\item \textsc{Versalita} (C-c-f-c).
\item \textsf{Serifa}, es decir, sin remates o paloseco (C-c-f-f).
\item \textrm{Romana} (C-c-f-r).
\end{itemize}


\section{Viñetas y enumerados}

En \LaTeX{} hay tres tipos básicos de viñetas:

\begin{itemize}
\item itemize.
\item enumerate.
\item description.
\end{itemize}


Es posible hacer viñetas (como la siguiente) cambiando márgenes u otras
propiedades gracias al paquete \href{http://mirror.ctan.org/macros/latex/contrib/enumitem/enumitem.pdf}{\emph{enumitem}}
(ya incluido en \arcopfc).

\begin{itemize}[noitemsep, label=$\triangleright$]
\item esto es
\item una pequeña
\item muestra
\end{itemize}

El paquete \emph{enumitem} ofrece muchas otras posibilidades para personalizar
las viñetas (individual o globalmente) o crear nuevas.


\section{Figuras}

Las figuras se referencian así (ver figura~\ref{fig:informatica}). Recuerda que
no tienen porqué aparecer en el lugar donde se ponen (mira un libro de
verdad). \LaTeX{} las colocará donde mejor queden, No te empeñes en
contradecirle, él sabe mucho de tipografía.

\begin{figure}[!h]
\begin{center}
\includegraphics[width=0.2\textwidth]{logos/emblema_informatica-gray.pdf}
\caption{Escudo oficial de informática}
\label{fig:informatica}
\end{center}
\end{figure}

Por cierto, los títulos de tablas, figuras y otro elementos flotantes (los
\texttt{caption}) no deben acabar en punto~\cite{sousa}.


\section{Cuadros}
\label{sec:uncuadro}

Se denominan «tablas» cuando contienen datos con relaciones numéricas. En
general se denominan «cuadros». Si las columnas están bien alineadas, las líneas
verticales estorban más que ayudan (no las pongas). Los cuadros se referencian
de este modo (ver cuadro~\ref{tab:rpc-semantics}).

\begin{table}[hp]
  \centering
  {\small
  


\begin{tabular}{p{.2\textwidth}p{.2\textwidth}p{.2\textwidth}p{.2\textwidth}}
  \tabheadformat
  \tabhead{Tipo de fallo}   &
  \tabhead{Sin fallos}      &
  \tabhead{Mensaje perdido} &
  \tabhead{Servidor caído}  \\
\hline
\textit{Maybe}         & Ejecuta:   1 & Ejecuta: 0/1        & Ejecuta: 0/1 \\
                       & Resultado: 1 & Resultado: 0        & Resultado: 0 \\
\hline
\textit{Al-least-once} & Ejecuta:   1 & Ejecuta:   $\geq$ 1 & Ejecuta:   $\geq$ 0 \\
                       & Resultado: 1 & Resultado: $\geq$ 1 & Resultado: $\geq$ 0 \\
\hline
\textit{At-most-once}  & Ejecuta:   1 & Ejecuta:   1        & Ejecuta: 0/1 \\
                       & Resultado: 1 & Resultado: 1        & Resultado: 0 \\
\hline
\textit{Exactly-once}  & Ejecuta:   1 & Ejecuta:   1        & Ejecuta:   1 \\
                       & Resultado: 1 & Resultado: 1        & Resultado: 1 \\
\hline
\end{tabular}


% Local variables:
%   coding: utf-8
%   ispell-local-dictionary: "castellano8"
%   TeX-master: "main.tex"
% End:

  }
  \caption[Semánticas de \acs{RPC} en presencia de distintos fallos]
  {Semánticas de \acs{RPC} en presencia de distintos fallos
    (\textsc{Puder}~\cite{puder05:_distr_system_archit})}
  \label{tab:rpc-semantics}
\end{table}


\section{Listados de código}
\label{sec:listado}

Puedes referenciar un listado con~\ref{code:hello}. Éste es un listado flotante,
pero también pueden ser «no flotantes» (mira la documentación del paquete
\href{http://www.ctan.org/get/macros/latex/contrib/listings/listings.pdf}{«listings»}).

\begin{listing}[
  float=ht,
  language = C,
  caption  = {«Hola mundo» en C},
  label    = code:hello]
#include <stdio.h>
int main(int argc, char *argv[]) {
    puts("Hola mundo\n");
}
\end{listing}

Puedes modificar el estilo por defecto para tus listados añadiendo un comando
\texttt{lstset} en tu \texttt{main.tex}. El código LaTeX del listado \ref{code:custom-listings}
añade un fondo gris claro y una línea en el margen izquierdo.

\begin{listing}[
  float=h!,
  caption  = {Personalizando los listados de código},
  label    = code:custom-listings]
\lstset{%
  backgroundcolor = \color{gray95},
  rulesepcolor    = \color{black},
}
\end{listing}



\section{Citas y referencias cruzadas}

Una cita~\cite{design_patterns} y una referencia a la segunda sección (véase
\S\,\ref{sec:uncuadro}).

Si estás viendo la versión PDF de este documento puedes pinchar la cita o el
número de sección. Son hiper-enlaces que llevan al elemento correspondiente. Todos los
elementos que hacen referencia a otra cosa (figuras, tablas, listados,
secciones, capítulos, citas, páginas web, etc.) pueden ser «pinchables» gracias al
paquete \href{http://latex.tugraz.at/_media/docs/hyperref.pdf}{«hyperref»}.

Para citar páginas web usa el comando \texttt{url} como en: \url{http://www.uclm.es}

\section{Páginas}
\label{sec:paginas}

La normativa dice que el documento debería ser impreso a una cara, pero si el
número de páginas es alto puede imprimirse a dos caras. Como eso es bastante
subjetivo, mi consejo es que ronde las 100 \textbf{hojas}. Es decir, si el
documento tiene más de 200 páginas imprímelo a doble cara, si tiene menos
imprímelo a una.

Por defecto, \arcopfc imprime a una cara (oneside), si quieres imprimir a doble cara,
escribe en el preámbulo:

\begin{listing}
  \documentclass[twoside]{arco-pfc}
\end{listing}

Esto es importante porque a doble cara los márgenes son simétricos y a una cara
no. Si llevas el PFC a la copistería y pides que te lo impriman de modo
diferente al generado quedará mal.


% Local Variables:
%   coding: utf-8
%   mode: latex
%   mode: flyspell
%   ispell-local-dictionary: "castellano8"
% End:
