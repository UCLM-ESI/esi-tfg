\documentclass{pre-tfg}

\showhelp  % comenta o borra para eliminar ayudas

\title{(título del TFG)}
\author{(nombre y apellidos)}
\advisorFirst{(nombre y apellidos)}
\advisorDepartment{(departamento del director)}
\advisorSecond{}
\intensification{(INTENSIFICACIÓN)}
\docdate{Año}{Mes}


\begin{document}

\maketitle
\tableofcontents

\newpage

El anteproyecto recogería, en un \textcolor[rgb]{0.5,0.0,0.0}{máximo de 10 páginas},
los siguientes apartados:

\begin{itemize}
\item Introducción (muy recomendable aunque no obligatorio)
\item Tecnología específica cursada por el alumno
\item Objetivos
\item Método y fases de trabajo
\item Medios que se pretenden utilizar
\item Bibliografía básica consultada en la elaboración del anteproyecto
\item Contrato de propiedad intelectual (si lo hubiera)
\end{itemize}


\section{INTRODUCCIÓN}

El capítulo de introducción podrá abordar los siguientes aspectos:

\begin{itemize}
\item Introducción al tema, entorno en el que el trabajo desempeñará
  su objetivo, justificación de la importancia del trabajo abordado.
\item Motivación y antecedentes (con algunas referencias bibliográficas).
\item Descripción gráfica del proyecto (es aconsejable incorporar una figura que describa
  el trabajo a desarrollar y que mejore la comprensión del mismo).
\end{itemize}


\section{TECNOLOGÍA ESPECÍFICA / INTENSIFICACIÓN / ITINERARIO CURSADO POR EL ALUMNO}

El Trabajo Fin de Grado (TFG, de ahora en adelante) siempre deberá demostrar la aplicación
de las competencias generales de la titulación. Además, el TFG deberá aplicar
\textbf{algunas} de las competencias específicas asociadas a la \textbf{Tecnología
  Específica o Intensificación} que el alumno ha cursado. Por lo tanto, el alumno incluirá
en el anteproyecto \textbf{dos tablas}. Una tabla para seleccionar la tecnología cursada y
en la que se contextualiza el TFG:

\begin{table}[hp]
  \centering
  \caption{Tecnología Específica cursada por el alumno}
  \label{tab:tec-especifica}

  \zebrarows{1}
  \begin{tabular}{p{0.6\textwidth}}
    \textbf{Marcar la tecnología cursada} \\
    \hline
    Tecnologías de la Información \\
    Computación \\
    Ingeniería del Software \\
    Ingeniería de Computadores \\
    \hline
  \end{tabular}
\end{table}


\clearpage

En la segunda tabla, el alumno deberá justificar cómo \textbf{algunas}
de las competencias específicas de la intensificación se aplicarán o
tomarán forma en el TFG, \textbf{La relación de competencias por
  intensificación se encuentran en el Anexo I al final de este
  documento. }


\begin{table}[hp]
  \centering
  \caption{Justificación de las competencias específicas abordadas en el TFG}
  \label{tab:competencias}

  \zebrarows{1}
  \begin{tabular}{p{0.2\linewidth}p{0.7\linewidth}}
    \textbf{Competencia} & \textbf{Justificación} \\
    \hline
    Competencia 1 & [Exponer y argumentar cómo y en qué parte se va a
    abordar esta competencia en el TFG]\\
    & \\
    & \\
    & \\
    \hline
  \end{tabular}
\end{table}


\section{OBJETIVOS}

De acuerdo a la Introducción, el alumno deberá especificar cuál o cuáles son las hipótesis
de trabajo de las que se parten, qué se pretende resolver, y en base a eso formular el
objetivo principal del TFG.

El objetivo principal deberá desglosarse en sub-objetivos parciales. Los sub-objetivos
deberán describirse de forma breve y concisa.

Como preámbulo a la formulación del objetivo parcial, el alumno deberá discutir sobre las
limitaciones y condicionantes a tener en cuenta en el desarrollo del TFG (lenguaje de
desarrollo, equipos, madurez de la tecnología, etc.).

Del mismo modo, será recomendable incluir una lista preliminar de requisitos del sistema a
construir.


\section{MÉTODO Y FASES DE TRABAJO}

Para el desarrollo del proyecto, el alumno deberá seguir algún proceso o metodología afín
al problema que pretende resolver. Para ello, deberá aportar una pequeña descripción del
proceso o metodología (no más de una página) y \textbf{justificar su adecuación al
  problema a resolver}.

Del mismo modo, el alumno podrá realizar una breve planificación de la ejecución del
proyecto según el proceso o metodología seleccionada.

Como parte de la descripción del método y las fases de trabajo, el alumno podrá incluir
una descripción preliminar de las tareas, una planificación temporal, diagramas de Gantt o
recursos similares que pueda considerar necesarios.

Si hubiera más de una metodología que a juicio del alumno podría ser afín al proyecto,
éstas deberán mencionarse, y justificar la que considera más adecuada (esto puede
considerarse parte de la justificación a la adecuación al problema a resolver).


\section{MEDIOS QUE SE PRETENDEN UTILIZAR}

\subsection{Medios Hardware}

El alumno deberá describir los medios hardware que prevé serán necesarios para el
desarrollo del proyecto.


\subsection{Medios Software}

El alumno deberá describir los medios software (lenguajes, entornos de desarrollo,
herramientas de gestión y planificación, etc.) que prevé serán necesarios para el
desarrollo del proyecto


\section{REFERENCIAS}

En esta sección se incluirán todas las referencias bibliográficas, ordenadas
alfabéticamente por el primer apellido del primer autor, de las obras de las cuales se
haya realizado alguna cita en los apartados anteriores. Las referencias deberán contener
datos básicos como nombre y apellidos de los autores, título de la obra, evento al que
pertenece, páginas, fecha y lugar de celebración (si se tratara de artículos de congreso),
ISBN, editorial y ciudad (si se tratara de libro), nombre de revista, páginas, volumen y
número (si se tratara de revista), etc.

Se empleará un formato de referencia reconocido en el ámbito académico como
ACM\footnote{http://www.acm.org/sigs/publications/proceedings-templates}\footnote{http://www.cs.ucy.ac.cy/\~{}chryssis/specs/ACM-refguide.pdf}.
Otros formatos aconsejables son, por ejemplo, IEEE, AMA, APA y AMA.

A continuación una sección de «Referencias» con ejemplos de referencias con formato ACM para:

\begin{itemize}
\item Un artículo de revista~\cite{Bow93}.
\item Un informe técnico~\cite{Ding97}.
\item Un libro~\cite{Tavel07}.
\item Un capítulo de libro~\cite{Greiner99}.
\item Un artículo en las actas de un congreso~\cite{Frohlic00}.
\item Para una página web~\cite{Steele04} (con autores conocidos).
\item Para una página web~\cite{Oxygen} (con autores desconocidos).
\end{itemize}


\bibliographystyle{alpha}
\singlespacing
\bibliography{main}

\section{CONTRATO DE PROPIEDAD INTELECTUAL (si lo hubiera)}

\newpage
\section*{ANEXO I: Descripción de Competencias por Intensificación o Tecnología
Específica\footnote{Este anexo se deberá borrar y no deberá ser incluido en el documento de anteproyecto final}}

\subsection*{Intensificación de Computación}

\begin{itemize}
\item Capacidad para tener un conocimiento profundo de los principios fundamentales y
  modelos de la computación y saberlos aplicar para interpretar, seleccionar, valorar,
  modelar, y crear nuevos conceptos, teorías, usos y desarrollos tecnológicos relacionados
  con la informática.
\item Capacidad para conocer los fundamentos teóricos de los lenguajes de programación y
  las técnicas de procesamiento léxico, sintáctico y semántico asociadas, y saber
  aplicarlas para la creación, diseño y procesamiento de lenguajes.
\item Capacidad para evaluar la complejidad computacional de un problema, conocer
  estrategias algorítmicas que puedan conducir a su resolución y recomendar, desarrollar e
  implementar aquella que garantice el mejor rendimiento de acuerdo con los requisitos
  establecidos.
\item Capacidad para conocer los fundamentos, paradigmas y técnicas propias de los
  sistemas inteligentes y analizar, diseñar y construir sistemas, servicios y aplicaciones
  informáticas que utilicen dichas técnicas en cualquier ámbito de aplicación.
\item Capacidad para adquirir, obtener, formalizar y representar el conocimiento humano en
  una forma computable para la resolución de problemas mediante un sistema informático en
  cualquier ámbito de aplicación, particularmente los relacionados con aspectos de
  computación, percepción y actuación en ambientes entornos inteligentes.
\item Capacidad para desarrollar y evaluar sistemas interactivos y de presentación de
  información compleja y su aplicación a la resolución de problemas de diseño de
  interacción persona computadora.
\item Capacidad para conocer y desarrollar técnicas de aprendizaje computacional y diseñar
  e implementar aplicaciones y sistemas que las utilicen, incluyendo las dedicadas a
  extracción automática de información y conocimiento a partir de grandes volúmenes de
  datos.
\end{itemize}


\subsection*{Intensificación de Ingeniería de Computadores}

\begin{itemize}
\item Capacidad de diseñar y construir sistemas digitales, incluyendo computadores,
  sistemas basados en microprocesador y sistemas de comunicaciones.
\item Capacidad de desarrollar procesadores específicos y sistemas empotrados, así como
  desarrollar y optimizar el software de dichos sistemas.
\item Capacidad de analizar y evaluar arquitecturas de computadores, incluyendo
  plataformas paralelas y distribuidas, así como desarrollar y optimizar software para las
  mismas.
\item Capacidad de diseñar e implementar software de sistema y de comunicaciones.
\item Capacidad de analizar, evaluar y seleccionar las plataformas hardware y software más
  adecuadas para el soporte de aplicaciones empotradas y de tiempo real.
\item Capacidad para comprender, aplicar y gestionar la garantía y seguridad de los sistemas informáticos.
\item Capacidad para analizar, evaluar, seleccionar y configurar plataformas hardware para
  el desarrollo y ejecución de aplicaciones y servicios informáticos.
\item Capacidad para diseñar, desplegar, administrar y gestionar redes de computadores.
\end{itemize}


\subsection*{Intensificación de Ingeniería del Software}

\begin{itemize}
\item Capacidad para desarrollar, mantener y evaluar servicios y sistemas software que
  satisfagan todos los requisitos del usuario y se comporten de forma fiable y eficiente,
  sean asequibles de desarrollar y mantener y cumplan normas de calidad, aplicando las
  teorías, principios, métodos y prácticas de la Ingeniería del Software.
\item Capacidad para valorar las necesidades del cliente y especificar los requisitos
  software para satisfacer estas necesidades, reconciliando objetivos en conflicto
  mediante la búsqueda de compromisos aceptables dentro de las limitaciones derivadas del
  coste, del tiempo, de la existencia de sistemas ya desarrollados y de las propias
  organizaciones.
\item Capacidad de dar solución a problemas de integración en función de las estrategias,
  estándares y tecnologías disponibles.
\item Capacidad de identificar y analizar problemas y diseñar, desarrollar, implementar,
  verificar y documentar soluciones software sobre la base de un conocimiento adecuado de
  las teorías, modelos y técnicas actuales.
\item Capacidad de identificar, evaluar y gestionar los riesgos potenciales asociados que pudieran presentarse.
\item Capacidad para diseñar soluciones apropiadas en uno o más dominios de aplicación
  utilizando métodos de la ingeniería del software que integren aspectos éticos, sociales,
  legales y económicos.
\end{itemize}


\subsection*{Intensificación de Tecnologías de la Información}

\begin{itemize}
\item Capacidad para comprender el entorno de una organización y sus necesidades en el
  ámbito de las tecnologías de la información y las comunicaciones.
\item Capacidad para seleccionar, diseñar, desplegar, integrar, evaluar, construir,
  gestionar, explotar y mantener las tecnologías de hardware, software y redes, dentro de
  los parámetros de coste y calidad adecuados.
\item Capacidad para emplear metodologías centradas en el usuario y la organización para
  el desarrollo, evaluación y gestión de aplicaciones y sistemas basados en tecnologías de
  la información que aseguren la accesibilidad, ergonomía y usabilidad de los sistemas.
\item Capacidad para seleccionar, diseñar, desplegar, integrar y gestionar redes e
  infraestructuras de comunicaciones en una organización.
\item Capacidad para seleccionar, desplegar, integrar y gestionar sistemas de información
  que satisfagan las necesidades de la organización, con los criterios de coste y calidad
  identificados.
\item Capacidad de concebir sistemas, aplicaciones y servicios basados en tecnologías de
  red, incluyendo Internet, web, comercio electrónico, multimedia, servicios interactivos
  y computación móvil.
\item Capacidad para comprender, aplicar y gestionar la garantía y seguridad de los sistemas informáticos.
\end{itemize}

\end{document}


% Local Variables:
% coding: utf-8
% mode: flyspell
% ispell-local-dictionary: "castellano8"
% mode: latex
% End:
